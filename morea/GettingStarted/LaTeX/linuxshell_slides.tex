%\documentclass[utf8x,10pt,handout]{beamer}
\documentclass[utf8x,10pt]{beamer}

%\usepackage{ics332}

\title{Linux Command Line Interface}

\begin{document}

\maketitle{}

\section{Foreword}

\begin{frame}
  \frametitle{Foreword}

  \begin{itemize}
  \item It is supposed to be a refresher (?!)
  \item If you are familiar with UNIX/Linux/MacOS X CLI, this is going
    to be boring...
  \item I will not talk about editors (vi, emacs...)
  \end{itemize}
  \vfill
\end{frame}

\section{Basics}
\subsection{Terminal. CLI. Shell}
\begin{frame}
  \frametitle{Basics}

  \begin{itemize}
  \item \textbf{(Computer) Terminal}: Hardware device for data entry and
    display;
  \item \textbf{Terminal Emulator} (aka tty): An application program
    replacing a computer terminal. Many of those, OS dependent (cmd
    for DOS/Win; Terminal for OS X; xterm for Linux; ...);
  \item The Terminal provides user access to the computer through the
    \textbf{Command Line Interface} (CLI) where the user issues commands.
  \item The CLI "dialect" is the \textbf{Shell}. In UNIX-like systems
    a lot of dialects exists: sh, bash, csh, tcsh, ksh, zsh...
  \item I will only use the \texttt{bash} shell.
  \end{itemize}
\end{frame}

\subsection{Shell}
\begin{frame}
  \frametitle{Shell}

  \begin{itemize}
  \item How to access a shell
    \begin{itemize}
    \item Logging in to your own Linux (virtual) box (CTRL-ALT-F1/CTRL-ALT-F7);
    \item Opening a graphic terminal (xterm...);
    \item SSHing into a server.
    \end{itemize}
  \item To know which shell you use: \texttt{echo \$SHELL}

    I use \texttt{this font} to denote commands
  \item SHELL is an environment variable
  \item \texttt{echo} is a command
  \end{itemize}
\end{frame}

\subsection{Commands}
\begin{frame}
  \frametitle{Commands}
  \scriptsize

  \begin{itemize}
  \item \texttt{apropos}: search the manual page names and descriptions
  \item \texttt{man}: manual page (try \texttt{man apropos}, \texttt{man man})
    
    Almost every command, system program, or API has a man page

    \texttt{man apropos}, \texttt{man fread}, \texttt{man pthreads},
    \texttt{man 1 open}, \texttt{man 2 open}

    Reading man pages is a very worthwhile activity

  \item Not everything is a command... \texttt{type <cmd>}
    
    \texttt{type man} (command), \texttt{type echo} (shell built-in),
    \texttt{type ls} (alias (well... on my system))
  \item Commands take arguments (optional or mandatory); take input
    from the standard input (aka \texttt{stdin}); produce output on
    the standard output (\texttt{stdout}) or the standard error output
    (\texttt{stderr}).

    \vfill

  \item UNIX philosophy: Have a lot of programs; Each of them does
    only one thing but does it well.
  \end{itemize}
\end{frame}

\begin{frame}
  \frametitle{First Commands}

  \begin{itemize}
  \item Print the current working directory: \texttt{pwd};
  \item List the files in the current directory: \texttt{ls};
  \item Create a directory: \texttt{mkdir <directoryName>};
  \item Change the current directory to \texttt{<directoryName>}:
    \texttt{cd <directoryName>};
  \item Create an empty file: \texttt{touch <fileName>};
  \end{itemize}
\end{frame}

\begin{frame}
  \frametitle{Wildcards}
  \begin{itemize}
  \item List the files whose names start with a \texttt{c}: \texttt{ls
      c*};
  \item List the files whose names end with a \texttt{c}: \texttt{ls
      *c};
  \item List the files whose names contain a \texttt{c}: \texttt{ls
      *c*};
  \item We could spend a whole session (and maybe more) on bash and
    the use of regular expressions (Look up \emph{bash regular
      expressions} with your favorite search engine)
  \end{itemize}
\end{frame}

\subsection{Pipes}
\begin{frame}
  \frametitle{UNIX Philosophy (continued) - Pipes}

  \emph{Expect the output of every program to become the input to
    another, as yet unknown, program. Don't clutter output with
    extraneous information. Avoid stringently columnar or binary input
    formats. Don't insist on interactive input}

  \begin{itemize}
  \item \texttt{grep}: find a string in a file or a set of files;

    \texttt{prompt> grep Prince TheLittlePrince.txt}

    \texttt{prompt> grep -i Prince TheLittlePrince.txt}

    ... [A lot]

  \item \texttt{wc}: count lines, words, characters

    \texttt{prompt> wc TheLittlePrince.txt}

    2582 17297 91651 TheLittlePrince.txt

    lines words characters
  \end{itemize}
\end{frame}

\begin{frame}
  \frametitle{UNIX Philosophy (continued) - Pipes}

  \begin{itemize}
  \item \texttt{$|$}: "pipe" (the output of the program at the left of the pipe
    becomes the input of the program at the right of the pipe)

    \texttt{prompt> grep -i Prince TheLittlePrince.txt $|$ wc -l}

    195
    
  \item and more... 

    \texttt{cat TheLittlePrince.txt $|$ sed 's/becuase/because/g'}

  \end{itemize}
\end{frame}

\section{Miscellaneous}
\subsection{Jobs Management}
\begin{frame}
  \frametitle{Jobs}

  \begin{itemize}
  \item Commands can be executed in the \emph{background} thanks to
    the \& symbol.

    \texttt{find / -type f | wc -l > /tmp/find.stdout 2> /tmp/find.stderr \&}

  \item The running command is called a \emph{job}
  \item \texttt{jobs} is the command to look at running jobs
  \item Jobs can be accessed as \%1, \%2, ...
  \item \texttt{fg \%2} brings the job \#2 to the \emph{foreground}
  \item If a job is already running, hitting CTRL-Z suspends the job
    and gives it a job id
  \item \texttt{bg \%4} resumes the suspended job in the background
  \item \texttt{kill \%7} attempts to terminate the job \#7 gracefully
  \item \texttt{kill -9 \%7} kills the job unconditionally
  \end{itemize}
\end{frame}

\subsection{Environment Variables}
\begin{frame}
  \frametitle{Environment Variables}

  \begin{itemize}
  \item \texttt{printenv} to get the list of environment variables;
  \item Sometimes you’ll have to set/modify environment variables
    (PATH) (and you can mess things up badly);
  \item Setting a new environment variable (or overwriting another
    one):

    \texttt{export NEWTHING="a:b/c"}
  \item Adding to an environment variable:

    \texttt{export NEWTHING="\$NEWTHING hello"}
  \end{itemize}
\end{frame}

\subsection{Shell Customization}
\begin{frame}
  \frametitle{Shell Customization}

  \begin{itemize}
  \item You are supposed to always know what you are doing

    \texttt{cp}, \texttt{rsync} can be useful for backups

  \item \texttt{chsh}: Change the shell
  \item Edit \texttt{.bashrc} and/or \texttt{.bash\_profile} to
    add/modify aliases, environment variables...

    \texttt{export PATH=\$PATH:\$HOME/bin}

    \hfill \emph{--- Tip: Always add the old path (unless you know what you are doing)}

    (Setting \texttt{export PATH=foo} is a bad idea... Why?)
  \item \texttt{source \$HOME/.bashrc} is faster than logging out and
    back in.
  \end{itemize}
\end{frame}

\subsection{Tricks and Tips}
\begin{frame}
  \frametitle{Tricks and Tips}

  \begin{itemize}
  \item Tab completion
  \item Up-arrow
  \item \texttt{history} and \texttt{!$<$something$>$}, \texttt{!ls}, \texttt{!!}, \texttt{!23}
  \item CTRL-R
  \item ... and others ...
  \end{itemize}
\end{frame}

\begin{frame}[fragile]
  \frametitle{sudo / System Updates/System Upgrades}

  \begin{itemize}
  \item \texttt{sudo} is dangerous (With great power comes great responsibility);
  \item Every now and then the system will tell you something like:
{\scriptsize
\begin{verbatim}
224 packages can be updated.
130 updates are security updates.

New release '16.04.1 LTS' available.
Run 'do-release-upgrade' to upgrade to it.
\end{verbatim}
}
  \item DO NOT \texttt{do-release-upgrade} your Virtual Machine during the semester
  \end{itemize}
\end{frame}

\section{Conclusion}
\subsection{Conclusion}
\begin{frame}
  \frametitle{Conclusion. Limits}

  \begin{itemize}
  \item We've barely scratched the surface...
  \item The CLI is very powerful
	\begin{itemize}
	\item But there are things it can't do (e.g., floating-point numbers)
	\end{itemize}
  \item Knowing a scripting language is a good idea for your future
    (bash, Python, Ruby?, Perl?...)
    \begin{itemize}
    \item Allows you to avoid tedious by-hand work (e.g., see the first "programming assignment")
	\begin{itemize}
    	\item There is a Scripting Languages course (ICS215)
	\end{itemize}
    \end{itemize}
  \end{itemize}
\end{frame}

\end{document}
